\documentclass{article}
\usepackage{graphicx} % Required for inserting images
\usepackage{url}

\title{Patents and Litigation in the Video Game Industry}
\author{Ryan Gallagher,
\\ Jam Wilder,
\\ Austin Snyder,`'
\\ Sean Gallaway}
\date{\today}

\begin{document}

\maketitle

\section{Introduction}
The video game industry is no stranger to wild legal battles. Just as there are patent battles and controversies in any other industry, there are many in the video game industry. What makes these special is that in so many cases, the legal process flies in the face of what the everyday consumer wants. In this paper, we explore three famous legal battles in which the masses can't enjoy widely popular games and mechanics. These cases being:
\begin{itemize}
    \item Nintendo vs PocketPair (The Palworld Lawsuit).
    \item Warner Bros filing a patent over the widely popular Nemesis System.
    \item PUBG vs Epic Games.
\end{itemize}
\par All of these cases have wide implications. PUBG suing Epic Games, those are two of the most popular video games currently, with Fortnite (Epic Games product which caused the lawsuit) being the most popular game in the world. PUBG is essentially suing because the gameplay is strikingly similar. The Nintendo lawsuit is also interesting, in this case, they are suing a similar game, Palworld, on the basis that they hold a patent for throwing a ball to catch creatures in a digital environment. The games are similar no doubt, but with such a broad patent will be interesting to see how the case develops. Perhaps the most infuriating patent case is that of Warner Bros with their Nemesis system. The Nemesis system involved defeating enemies, who then learn the player's patterns and behaviors, and adapt to that for subsequent encounters. It was widely beloved, but now sits idly by until the year 2036 while it is under patent.
\par Common to all of these situations, the consumer is being deprived of a fun experience by the patent laws which are in place. There is an interesting argument to be had about fair use, and taking inspiration from popular works to make a new product which incorporates functionalities which people loved in prior games.


















\section{Wider Issues in the Software World}
These examples are simply ones in which consumers are being deprived of fun experiences in the video game industry. There are many other cases of very similar issues around the wider software world, not just in entertainment, but also products which consumers interact with in their everyday life. Or, even worse, a software company taking advantage of their consumers. There have been many examples thoughout the history of the software industry of this. In fact, it is quite a common issue right now with large language models, since the companies developing them need so much data. They are being accused of copywrite infringement, and theft of intellectual property by many. Worse than this, we may see more cases where companies are outright harming their consumers, as we will discuss with Sony BMG.

\subsection{Soverain Software}
Soverain Software is commonly known as a "patent troll", maintaining many patents not for the purposes of business use, but rather just to claim royalties from others. An everyday example, Soverain Software at one time held a patent for the "digital shopping cart". A digital shopping cart is a feature which is available on most e-commerce websites today, but that may have been impossible without the case we are going to discuss. Soverain Software sued and settled with Amazon, and Gap inc. in 2010, In the U.S. District Court for the Eastern District of Texas. Gap settled without disclosing the terms, but Amazon settled for \$40 million \cite{Soverain}. 

After a few more years of suing various other companies for infringing on their digital shopping cart patent, Soverain Software finally was defeated in 2013 in the U.S Court of Appeals for the Federal Circuit. Where several of their judgements were reversed, as being "plain in the view of the prior art", or in other words, people had done similar things before, so it wasn't new or inventive \cite{Soverain}. The U.S. Supreme Court denied Soverain Software's petition to hear their case, and as such the ruling from the Federal Court against Soverain Software stood. Their patents were invalidated, and their lawsuits were killed \cite{Soverain}. 

This is a case where the consumer gets a better product, and the software industry prevails. One company doesn't get to hold a patent over a very common practice, and users get to use a simple feature universally across many websites. 

\subsection{Sony BMG Copy Protection Scandal}
In 2005, information came out that Sonyn BMG had been installing one of two pieces of software on consumer's machines, unbeknownst to the consumers. This software was actually bunlded directly onto CD's themselves, so a user had no way of expecting this to happen \cite{sony}. These softwares provided digital rights management, they would modify the user's operating system to interfere with copying CD's \cite{sony}. Neither of these two programs could easily be removed from the user's machine as well. 

A large issue with this, these two programs that Sony was installing on people's machines were introducing security vulnerabilities, taking system resources, and even causing crashes. Not only were they changing your machine without any sort of permission, they were also installing a program which jeopardized your security. Sony BMG eventually released an uninstaller to get these programs off of a user's machine. In fact, this uninstaller simply made the files invisible, and didn't uninstall anything \cite{sony}. Of course, it also installed additional software which could not be removed easily, and brought on even more security issues \cite{sony}. 

Sony BMG eventually ended up paying out several class action lawsuits on the matter. This is a great example of a software company pushing the envelope as far as what they can get away with doing to their consumers. What is quite shocking is how Sony BMG doubled down, and lied to the masses again, simply making files invisible, and then creating even \textbf{more} issues on user machines. 


\section*{References}

\noindent
\textbf{\cite{Palworld}} Julia Alexander. *Palworld removes feature following Nintendo lawsuit speculation*. The Verge, 2024. \\
\url{https://www.theverge.com/news/663210/palworld-updates-feature-removed-nintendo-lawsuit}

\vspace{0.5em}

\noindent
\textbf{\cite{Nemesis}} Engadget Staff. *Shadow of Mordor’s innovative Nemesis system is locked behind a patent until 2036*. Engadget, 2021. \\
\url{https://www.engadget.com/gaming/shadow-of-mordors-innovative-nemesis-system-is-locked-behind-a-patent-until-2036-195437208.html}

\vspace{0.5em}

\noindent
\textbf{\cite{PUBGvsFortnite}} Red Points. *PUBG sues Fortnite: A copyright battle royale*. Red Points Blog, 2022. \\
\url{https://www.redpoints.com/blog/pubg-sues-fortnite-a-copyright-battle-royale/}

\noindent
\textbf{\cite{Soverain}} Wikipedia. *Soverain Software*. Wikipedia, 2024. \\
\url{https://en.wikipedia.org/wiki/Soverain_Software?utm_source=chatgpt.com}

\noindent
\textbf{\cite{sony}} Wikipedia. *Sony BMG copy protection rootkit scandal*. Wikipedia, 2024. \\
\url{https://en.wikipedia.org/wiki/Sony_BMG_copy_protection_rootkit_scandal?utm_source=chatgpt.com}

\end{document}
